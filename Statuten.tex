\documentclass[12pt]{article}
\usepackage[a4paper, headheight=2.5cm, top=4.5cm, bottom=2.5cm, left=2.5cm, right=2.5cm]{geometry}
\usepackage[utf8]{inputenc}
\usepackage[T1]{fontenc}
\usepackage[ngerman]{babel}
\usepackage{graphicx}
\usepackage{enumitem}
\usepackage{fancyhdr}
\usepackage{hyperref}
\usepackage{microtype}
\usepackage{amsmath}
\usepackage{tocloft}
\usepackage{titlesec}
\usepackage{lastpage}
\usepackage{tabularx}
\usepackage{makecell}
\usepackage{ragged2e}
\usepackage{eurosym}

\renewcommand{\headrulewidth}{0pt}
\titleformat{\section}{\large\bfseries}{§ \thesection}{1em}{}
\renewcommand{\cftsecleader}{\cftdotfill{\cftdotsep}}

\title{Vereinssatzung \\
ARTandTECH.space e.V.}
\author{
	Lindenstraße 11 \\
	48431 Rheine \\
    \
	}

\hypersetup{
  pdftitle={Vereinssatzung ARTandTECH.space e.V.},
  pdfauthor={ARTandTECH.space e.V.},
  pdfsubject={Vereinssatzung},
  pdfkeywords={Verein, Satzung, Rheine},
  colorlinks=false,
  pdfborder={0 0 0},
  pdfstartview=Fit,
}

\date{24. April 2024}

\begin{document}

% Titelseite
\maketitle
\thispagestyle{empty}

\begin{center}
	\includegraphics[width=0.3\textwidth]{assets/logo.png}
\end{center}


\thispagestyle{fancy}

\begin{abstract}
	Der ARTandTECH.space ist ein gemeinsames Projekt der Stadt Rheine und des Kreises Steinfurt.
	Hier wird die vorwiegend naturwissenschaftlich technisch orientierte MINT-Bildung, der kreative
	Umgang mit Technik, mit der kulturellen, ästhetischen und künstlerischen Bildung verknüpft. So
	entstehen Chancen für die Menschen auf Partizipation, Identitätsbildung und die Entwicklung
	elementarer sozialer Kompetenzen wie Teamfähigkeit und Verantwortungsbewusstsein.
	Eine offene und einladende Atmosphäre bietet niedrigschwellige Zugänge für jeden, der den
	ARTandTECH.space im Sinne der Vereinsziele nutzen möchte und/oder durch seine Mitarbeit
	unterstützt. Die Mitglieder und Nutzer:innen pflegen einen solidarischen Umgang, der jedem gleiche
	Chancen für eine Teilhabe ermöglicht.
\end{abstract}
\newpage

% Inhaltsverzeichnis
\pagestyle{fancy}
\fancyhf{}
\fancyhead[L]{\textbf{Vereinssatzung}}
\fancyhead[R]{\includegraphics[width=0.15\textwidth]{assets/logo.png}}
\tableofcontents
\thispagestyle{fancy}
\newpage

% Inhalt
\fancyfoot[R]{Seite \thepage\ von \pageref{LastPage}}
\setcounter{page}{1}

\section{Name, Sitz, Geschäftsjahr}
% Numberd Item
\begin{enumerate}[label=(\arabic*)]
	\item Der Verein führt den Namen ARTandTECH.space e.V. Er wird in das Vereinsregister in Rheine eingetragen und führt danach den Zusatz „e.V.".
	\item Der Verein hat seinen Sitz in Rheine
	\item Das Geschäftsjahr ist das Kalenderjahr. Das erste Geschäftsjahr beginnt mit der Eintragung des Vereins in das Vereinsregister und endet am 31.12. desselben Jahres.
\end{enumerate}
\section{Zweck des Vereins}

\begin{enumerate}[label=(\arabic*)]
	\item Der Verein verfolgt ausschließlich und unmittelbar gemeinnützige Zwecke im Sinne des
	      Abschnitts "Steuerbegünstigte Zwecke" der Abgabenordnung (AO).
	\item Zweck des Vereins ist die Förderung von Wissenschaft und Forschung, Jugendhilfe, Kunst und
	      Kultur, Bildung und Erziehung sowie bürgerschaftlichen Engagements im Sinne des § 52 Abs. 2 AO.
	      Der Satzungszweck wird insbesondere verwirklicht durch die interdisziplinäre Förderung der
	      kulturellen Bildung und der MINT-Bildung. Hierzu gehören die Planung, Unterstützung und
	      Durchführung von kreativen und technischen Projekten, die Beschaffung von Finanzmitteln sowie
	      dem Aufbau, Einrichtung und Betrieb von Kreativ- und Arbeitsräumen sowie von Werkstätten für den
	      ARTandTECH.space e.V. in Rheine.
	\item Daneben kann der Verein seinen Förderzweck auch verwirklichen, in dem er MINT-Projekte
	      sowohl als auch Projekte der kulturellen Bildung unterstützt durch:
	      \begin{itemize}
		      \item die Bereitstellung von Geräten, Betriebsmitteln, Materialien, Arbeitsplätzen.
		      \item eine individuelle Begleitung von Projekten sowohl auf fachlicher wie auf organisatorischer Ebene.
		      \item die Unterstützung bei Teilnahmen an Wettbewerben und Veranstaltungen der kulturellen und
		            MINT-Bildung.
		      \item die Organisation eigener Kurse, Angebote, Aktionen und Veranstaltungen für Menschen jeglichen
		      \item die Vernetzung und Kooperation von Akteuren, die ähnliche Ziele verfolgen.
		      \item Weiterbildungsmaßnahmen intern wie extern.
	      \end{itemize}

\end{enumerate}

\section{Gemeinnützigkeit}
\begin{enumerate}[label=(\arabic*)]
	\item Der Verein ist selbstlos tätig; er verfolgt nicht in erster Linie eigenwirtschaftliche Zwecke.
	\item Alle Mittel des Vereins dürfen nur für die satzungsmäßigen Zwecke verwendet werden.
	\item Es darf keine Person durch Ausgaben, die dem Zweck der Körperschaft fremd sind, oder durch
	      unverhältnismäßig hohe Vergütungen begünstigt werden.
	\item Mitglieder des Vereins nehmen ihre Aufgabe ehrenamtlich wahr und erhalten in dieser
	      Eigenschaft keine Zuwendungen aus Mitteln des Vereins. Nachgewiesene Aufwendungen,
	      insbesondere Reisekosten, können erstattet werden.
\end{enumerate}

\section{Erwerb der Mitgliedschaft}
\begin{enumerate}[label=(\arabic*)]
	\item Mitglieder können natürliche und juristische Personen werden.
	\item Die Mitgliedschaft ist schriftlich zu beantragen. Über die Annahme entscheidet der Vorstand nach
	      freiem Ermessen. Die Mitgliedschaft beginnt mit der Annahme des Antrags. Gegen die Ablehnung,
	      die keiner Begründung bedarf, steht der/dem Bewerber:in die Berufung an die
	      Mitgliederversammlung zu. Diese entscheidet abschließend.
	\item Die Mitgliederversammlung kann Personen, die sich besondere Verdienste um den Verein oder
	      um die von ihm verfolgten satzungsgemäßen Zwecke erworben haben, zu Ehrenmitgliedern
	      ernennen. Ehrenmitglieder haben alle Rechte eines ordentlichen Mitglieds. Sie sind von
	      Beitragsleistungen befreit.
	\item Weiteres kann eine Mitgliederordnung regeln, die durch die Mitgliederversammlung beschlossen
	      wird.
\end{enumerate}

\section{Ende der Mitgliedschaft}
\begin{enumerate}[label=(\arabic*)]
	\item Die Mitgliedschaft endet durch
	      \begin{itemize}
		      \item Tod,
		      \item freiwilligen Austritt,
		      \item Streichung von der Mitgliederliste,
		      \item Ausschluss,
		      \item Verlust der Rechtsfähigkeit
	      \end{itemize}
	\item Der freiwillige Austritt erfolgt durch schriftliche Erklärung gegenüber dem Vorstand. Er ist nur zum
	      Ende eines Quartals mit Frist von 4 Wochen möglich.
	\item Die Streichung von der Mitgliederliste erfolgt durch Beschluss des Vorstands, wenn das Mitglied
	      trotz zweimaliger Mahnung mit der Zahlung des Beitrages in Rückstand ist oder der Rückstand mehr
	      als einen Jahresbeitrag beträgt. Die Streichung der Mitgliederliste darf erst dann beschlossen
	      werden, wenn nach Absendung der Mahnung ein Monat verstrichen ist und die Streichung angedroht
	      wurde.
	\item Der Vorstand kann einen Ausschluss beschließen.
	      Gründe hierfür sind insbesondere
	      \begin{itemize}
		      \item ein die Vereinsziele schädigendes Verhalten,
		      \item die Verletzung satzungsmäßiger Pflichten.
	      \end{itemize}
	      Die Einleitung des Ausschlussverfahrens sowie die Entscheidung darüber werden dem Mitglied in
	      einfacher Briefform mitgeteilt. Der Beschluss über den Ausschluss ist mit einer Mehrheit von zwei
	      Dritteln des Vorstands zu fassen. Vor der Beschlussfassung ist dem Mitglied unter Einhaltung einer
	      Mindestfrist von 10 Tagen Gelegenheit zu geben, sich persönlich zu rechtfertigen. Eine schriftliche
	      Stellungnahme ist möglich.
	      Gegen den Ausschluss steht dem Mitglied die Berufung an die Mitgliederversammlung zu. Die
	      Berufung ist schriftlich, innerhalb von 30 Tagen nach Zugang des Ausschlusses, an den Vorstand zu
	      richten. Die Berufung hat keine aufschiebende Wirkung. Die Mitgliederversammlung entscheidet mit
	      der Mehrheit von zwei Dritteln der anwesenden Mitglieder endgültig über den Ausschluss. Der
	      Rechtsweg vor einem ordentlichen Gericht ist ausgeschlossen.
\end{enumerate}

\section{Mitgliedsbeiträge}
\begin{enumerate}[label=(\arabic*)]
	\item Der Verein erhebt von den Mitgliedern Beiträge. Das Nähere regelt eine Beitragsordnung, die
	      insbesondere folgende Regelungen beinhaltet:
	      \begin{itemize}
		      \item Aufnahmebeitrag
		      \item Mitgliedsbeiträge in Form von Jahresbeiträgen
		      \item Zahlungsmodalitäten
		      \item Mahnkosten.
	      \end{itemize}
	      Die Beitragsordnung wird von der Mitgliederversammlung beschlossen.
\end{enumerate}

\section{Organe des Vereins}
\begin{enumerate}[label=(\arabic*)]
	\item Die Organe des Vereins sind die Mitgliederversammlung und der Vorstand.
	\item Der Verein kann Beiräte haben, die den Vorstand bei seinen Entscheidungen beraten.
\end{enumerate}

\section{Mitgliederversammlung}
\begin{enumerate}[label=(\arabic*)]
	\item Die Mitgliederversammlung ist das oberste Organ des Vereins. Eine ordentliche
	      Mitgliederversammlung ist vom Vorstand mindestens einmal im Kalenderjahr einzuberufen. Die
	      Einladung erfolgt mit einer Frist von mindestens vier Wochen. Für außerordentliche
	      Mitgliederversammlungen gilt eine Frist von zwei Wochen. Die Einladung erfolgt per E-Mail und durch
	      Aushang in den Vereinsräumen. Dabei sind den Mitgliedern die Tagesordnung bekannt zu geben und
	      die nötigen Informationen zugänglich zu machen.
	\item Mitgliederversammlungen finden grundsätzlich als Präsenzversammlungen statt.
	      Der Vorstand kann jedoch beschließen, dass die Mitgliederversammlung ausschließlich als virtuelle
	      Mitgliederversammlung in Form einer onlinebasierten Versammlung (virtuelle llversammlung) oder als
	      Kombination von Präsenzversammlung und virtueller Versammlung (hybride Vollversammlung)
	      stattfindet. Die teilnahmeberechtigten Personen haben keinen Anspruch darauf, virtuell an einer
	      Mitgliederversammlung teilzunehmen, die als Präsenzversammlung durchgeführt wird.
	\item Teilnahme- und stimmberechtigten Personen wird im Falle der Durchführung einer virtuellen
	      Mitgliederversammlung durch geeignete technische Vorrichtungen die Möglichkeit gegeben, online
	      an der Versammlung teilzunehmen und das Stimmrecht auf elektronischem Wege auszuüben.
	      Gleiches gilt im Falle der Durchführung einer hybriden Mitgliederversammlung für die teilnahme- und
	      stimmberechtigten Personen, die nicht in Präsenzform an der Versammlung teilnehmen.
	\item Die Auswahl der technischen Rahmenbedingungen (z. B. die Auswahl der zu verwendenden
	      Software bzw. Programme) obliegt dem Vorstand.
	\item Technische Widrigkeiten, die zu einer Beeinträchtigung bei der Teilnahme und bei der
	      Stimmrechtsausübung führen, berechtigen die teilnahme- und stimmberechtigten Personen nicht
	      dazu, gefasste Beschlüsse und vorgenommene Wahlen anzufechten, es sei denn, die Ursache der
	      technischen Widrigkeiten ist dem Verantwortungsbereich des Vereins zuzurechnen.
	      Im Übrigen gelten für die virtuellen und die hybriden Mitgliederversammlungen die
	      Vorschriften über die Mitgliederversammlungen sinngemäß.
	\item Die Mitgliederversammlung beschließt über
	      \begin{itemize}
		      \item alle in dieser Satzung genannten Ordnungen, soweit sie nicht einem anderen Organ zugewiesen
		      \item die ihr mit dieser Satzung zugewiesenen Aufgaben
		      \item die Entlastung des Vorstandes,
		      \item die Wahl und Abwahl des Vorstandes,
		      \item die Entgegennahme der Berichte des Vorstandes,
		      \item die Wahl der Kassenprüfer:innen
		      \item die Entgegennahme des Berichts der Kassenprüfer:innen
		      \item Genehmigung der Jahresrechnung für das abgelaufene Geschäftsjahr
		      \item Beschlussfassung über die Änderungen der Satzung und über die Auflösung des Vereins
		      \item Festsetzung von Beiträgen
		      \item Ernennung von Ehrenmitgliedern
		      \item Beschlussfassung über Anträge
		      \item weiteren Ordnungen zum Betrieb und Abläufen innerhalb des Vereins
	      \end{itemize}
	      Ausgenommen sind Satzungsänderungen, die von Aufsichts-, Gerichts- und/oder Finanzbehörden
	      aus formalen oder sonstigen Gründen verlangt werden. Über solche Satzungsänderungen
	      entscheidet der Vorstand.
	      Überdies entscheidet die Mitgliederversammlung über die Angelegenheiten, die ihr angetragen
	      werden.
	\item Der Vorstand beruft außerordentliche Mitgliederversammlungen ein, wenn das Interesse des
	      Vereins dies erfordert oder wenn die Einberufung von einem Zehntel der Mitglieder schriftlich unter
	      Angabe der gewünschten Tagesordnungspunkte verlangt wird. Im Übrigen gelten die gleichen
	      Verfahrensweisen wie bei ordentlichen Mitgliederversammlungen.
	\item Die Mitgliederversammlung wird von einem Vorstandsmitglied geleitet.
	\item Jede satzungsgemäß einberufene Mitgliederversammlung wird als beschlussfähig ohne Rücksicht
	      auf die Zahl der erschienenen Mitglieder anerkannt.
	\item Jedes mindestens 14 Jahre alte Mitglied hat genau eine Stimme. Das Stimmrecht kann nur
	      persönlich ausgeübt werden. Mitglieder unter 14 Jahren nehmen als beratende Mitglieder teil.
	\item Jedes Mitglied hat das Recht, Anträge zur Beschlussfassung in die Mitgliederversammlung
	      einzubringen. Diese können bis zu einer Woche vor dem Sitzungstermin eingebracht werden. Der
	      Vorstand informiert die Mitglieder unverzüglich über etwaige neue Anträge. Der Vorstand kann
	      Anträge gegebenenfalls auch ohne diese Frist zulassen.
	\item Alle Beschlüsse der Mitgliederversammlung werden mit einfacher Stimmenmehrheit gefasst,
	      soweit gesetzlichen Regelungen oder diese Satzung nichts Anderes vorschreiben. Bei
	      Stimmengleichheit entscheidet die Stimme des Versammlungsleiters. Stimmenthaltungen gelten als
	      nicht herausgegebene Stimme.
	\item Über die Beschlüsse der Mitgliederversammlung und, soweit zum Verständnis über deren
	      Zustandekommen erforderlich, auch über den wesentlichen Verlauf der Verhandlung, ist eine
	      Niederschrift zu fertigen. Hierzu bestellt die Mitgliederversammlung einen Schriftführer.
	      Die Niederschrift ist vom Schriftführer und dem Vereinsvorsitzenden zu unterschreiben und den
	      Mitgliedern zeitnah per E-Mail zu übersenden.
	\item Mitgliederversammlungen sind nicht öffentlich. Es können Gäste zugelassen werden. Über ihre
	      Teilnahme entscheidet die Mitgliederversammlung.
\end{enumerate}

\section{Vorstand}
\begin{enumerate}[label=(\arabic*)]
	\item Der Vorstand besteht aus:
	      Der/Dem Vorstandsvorsitzenden, fünf Stellvertretern:innen der/dem Kassierer:in sowie bis zu vier
	      Beisitzer:innen.
	\item Geschäftsführender Vorstand im Sinne des § 26 BGB sind:
	      Der/Die Vorstandsvorsitzende, fünf Stellvertretern:innen sowie der/dem Kassierer:in.
	      Sie vertreten den Verein gerichtlich und außergerichtlich. Zwei Mitglieder des geschäftsführenden
	      Vorstand vertreten gemeinsam.
	      Beim Abschluss von Grund- oder Vermögenserwerb, Kreditverträgen und laufenden finanziellen
	      Verpflichtungen oder Arbeitsverträgen vertritt die/der Kassenwart:in mit einem weiteren Mitglied des
	      geschäftsführenden Vorstands (Vorsitzende:n oder Stellvertreter:in).
	\item Die Vereinigung mehrerer Vorstandsämter in einer Person ist nicht zulässig.
	\item Die Mitglieder des Vorstandes werden von der Mitgliederversammlung gewählt, soweit im
	      Folgenden nichts Anderes geregelt ist.
	      Die Vorstandsmitglieder sind ehrenamtlich tätig. Für die Beisitzer:innen gilt ein Mindestalter von 14
	      Jahren, die übrigen Vorstandsmitglieder müssen volljährig sein.
	      Ein Vorstandsmitglied kann als Vertreter:in der/des Kassenwartes:wartin benannt werden.
	      Vorstandsmitgliedern kann durch Beschluss des Vorstands ein besonderer Geschäftsbereich
	      zugewiesen werden.
	      Es können nur Mitglieder des Vereins Vorstandsmitglied werden. Alle Vorstandsmitglieder sind
	      stimmberechtigt.
	\item Geborene Mitglieder:
	      Je eine von der Stadt Rheine und dem Kreis Steinfurt benannte Person ist geborenes Mitglied im
	      Vorstand. Sie muss nicht Mitglied des Vereins sein. Sie nimmt im Vorstand mindestens das Amt
	      einer/s Stellvertreters:in wahr.
	\item Der Vorstand wird auf die Dauer von zwei Jahren gewählt und bleibt bis zur Neuwahl im Amt. Die
	      Wiederwahl ist zulässig.
	\item Der Vorstand ist für alle Angelegenheiten des Vereins zuständig, soweit diese nicht durch die
	      Satzung einem anderen Organ des Vereins zugewiesen sind. Er hat insbesondere folgende
	      Aufgaben:
	      - Vorbereitung der Mitgliederversammlung und Aufstellung der Tagesordnung,
	      - Einberufung der Mitgliederversammlung,
	      - Ausführung der Beschlüsse der Mitgliederversammlung und
	      - Aufstellung eines Haushaltsplanes für jedes Geschäftsjahr sowie die Buchführung und Erstellung
	      eines Jahresberichtes.
	      Der Vorstand kann sich zur Erledigung seiner Aufgaben eine Geschäftsordnung geben. Diese wird in
	      der nächsten Mitgliederversammlung bekannt gegeben.
	\item Mit Beendigung der Vereinsmitgliedschaft endet auch die Mitgliedschaft im Vereinsvorstand.
	      Scheidet ein Mitglied des Vorstandes vorzeitig aus, so können die verbleibenden Vorstandsmitglieder
	      für die restliche Amtsdauer einen Nachfolger bestimmen.
	\item Der Vorstand fasst seine Beschlüsse in Sitzungen mit einfacher Mehrheit der abgegebenen
	      Stimmen. Der Vorstand ist beschlussfähig, wenn mindestens die Hälfte seiner Mitglieder anwesend
	      ist. Bei Stimmengleichheit entscheidet die Stimme des Vorsitzenden.
	\item Die/Der Vorsitzende beruft die Vorstandssitzungen ein und leitet sie. Die Beschlüsse des
	      Vorstandes sind zu protokollieren.
	\item Der Vorstand kann seine Beschlüsse auch im schriftlichen oder mündlichen Verfahren fassen,
	      sofern kein Vorstandsmitglied widerspricht.
\end{enumerate}

\section{Vergütung der Tätigkeit, externe Mitarbeiter und Geschäftsführung}
\begin{enumerate}[label=(\arabic*)]
	\item Der Vorstand kann bei Bedarf und unter Berücksichtigung der wirtschaftlichen Verhältnisse und
	      der Haushaltslage beschließen, dass Vereins- und Organämter entgeltlich auf der Grundlage eines
	      Dienst- oder Arbeitsvertrages oder gegen Zahlung einer pauschalen Aufwandsentschädigung gem. §
	      3 Nr. 26 a EStG ausgeübt werden. Für die Entscheidung über Vertragsbeginn, Vertragsinhalte und
	      Vertragsende ist der Vorstand zuständig.
	      Der Vorstand kann bei Bedarf und unter Berücksichtigung der wirtschaftlichen Verhältnisse und der
	      Haushaltslage ebenfalls Aufträge über Tätigkeiten für den Verein gegen eine angemessene
	      Vergütung oder Honorierung an Dritte vergeben.
	\item Zur Erledigung der Geschäftsführungsaufgaben und zur Führung der Geschäftsstelle ist der
	      Vorstand ermächtigt, im Rahmen der wirtschaftlichen Verhältnisse und der Haushaltslage eine
	      Geschäftsstellenleitung und/oder Mitarbeiter:innen für die Verwaltung einzustellen. Die
	      Geschäftsstellenleitung ist verpflichtet, an den Sitzungen des Vorstandes mit beratender Stimme
	      teilzunehmen. Die arbeitsrechtliche Direktionsbefugnis hat der 1. Vorsitzende.
	      Alternativ kann auf vorherigen Beschluss der Mitgliederversammlung nach §27 BGB
	      vertretungsberechtigte Vorstandsmitglieder auch als geschäftsführenden Vorstandsmitglieder
	      gewählt und auf Vergütungsbasis tätig werden. Ihnen kann auf Beschluss der Mitgliederversammlung
	      Befreiung von den Einschränkungen des §181 BGB erteilt werden.
\end{enumerate}

\section{Beirat}
\begin{enumerate}[label=(\arabic*)]
	\item Der Verein kann Beiräte haben. Diese sind ausschließlich beratend tätig und nicht
	      entscheidungsbefugt. Die Mitglieder werden durch den Vorstand oder die Mitgliederversammlung
	      berufen.
	\item Mitglieder der Beiräte müssen nicht Mitglieder des Vereins sein. Die Mitgliederversammlung ist
	      über die Berufung zu informieren. Sie kann Beiratsmitglieder mit mehr als 2/3 der anwesenden
	      Mitglieder abberufen.
	      Die Amtszeit eines Beiratsmitglieds beträgt ein Jahr. Eine Wiederbestellung nach Ablauf der
	      jeweiligen Amtszeit ist möglich. Beiratsmitglieder können ihr Amt jederzeit durch schriftliche Erklärung
	      an den Vorstand niederlegen. Die Mitgliederversammlung ist über Veränderungen in den Beiräten zu
	      informieren.
	\item Aufgabe der Beiräte ist es, den Vorstand bei der Durchführung seiner Aufgaben zu unterstützen
	      und den Vereinszweck und die Ziele zu fördern.
	\item Die Beiräte wählen aus ihrem Kreis
	      - die Beiratsvorsitzende bzw. den Beiratsvorsitzenden,
	      - die stellvertretende Beiratsvorsitzenden bzw. den stellvertretenden Beiratsvorsitzenden
	      für die Dauer von einem Jahr. Die Wiederwahl ist zulässig. Die bzw. der Beiratsvorsitzende und/oder
	      die bzw. der stellvertretende Beiratsvorsitzende bleiben jedoch nach Ablauf ihrer jeweiligen Amtszeit
	      so lange kommissarisch im Amt, bis eine Nachfolgerin bzw. ein Nachfolger gewählt ist.
	\item Die Beiräte treten in der Regel mindestens zweimal jährlich auf Einladung der/des
	      Beiratsvorsitzenden zusammen.
	\item Bei Bedarf können weitere Sitzungen der Beiräte von der Beiratsvorsitzenden einberufen werden.
	      Die bzw. der Vorsitzende hat den Beirat auch dann einzuberufen, wenn mindestens ein
	      Vereinsvorstandsmitglied oder wenigstens zwei Mitglieder des Beirates es verlangen.
	\item Die Mitglieder des Beirates üben ihre Tätigkeit ehrenamtlich aus. Aufwendungen, insbesondere
	      Reisekosten, werden auf Antrag erstattet.
	\item Der Beirat kann sich eine Geschäftsordnung geben.
\end{enumerate}

\section{Kassenprüfung}
\begin{enumerate}[label=(\arabic*)]
	\item Mindestens mit Abschluss des Geschäftsjahres ist die Kasse des Vereins zu prüfen. Hierzu wählt
	      die Mitgliederversammlung zwei Kassenprüfer:innen. Die Aufgabe wird für die Dauer von zwei Jahren
	      wahrgenommen, wobei ein:e Kassenprüfer:in jedem Geschäftsjahr zu wählen ist. Dies bedeutet,
	      dass im Jahr der Vereinsgründung ein:e Kassenprüfer:in nur für eine Amtszeit von einem Jahr zu
	      wählen ist. Eine Wiederwahl nach Ende der Amtszeit ist zulässig.
	\item Die Kassenprüfer:innen dürfen nicht Mitglieder des Vorstands sein.
	\item Anstelle der Kassenprüfer:innen kann die Mitgliederversammlung auch einen Wirtschaftsprüfer
	      mit der jährlichen Kassenprüfung beauftragen.
\end{enumerate}

\section{Haftung}
Der Verein haftet nicht für Schäden oder Verluste die Mitglieder bei Benutzung von Anlagen und
Einrichtungen des Vereins oder bei Vereinsveranstaltungen erleiden, soweit solche Schäden nicht
durch Versicherungen gedeckt sind. § 276 BGB bleibt unberührt.

\section{Kommunikationswege im Verein}
\begin{enumerate}[label=(\arabic*)]
	\item  Soweit in dieser Satzung nicht anders festgelegt ist, werden alle Informationen, Einladungen,
	      Niederschriften ausschließlich elektronisch versandt bzw. zur Verfügung gestellt. Dies gilt auch für
	      die Einladungen zu Mitgliederversammlungen, Zahlungserinnerungen und Mahnungen.
	\item Die unter (1) genannten Bekanntmachungen können zusätzlich durch Aushang an geeigneten
	      Stellen veröffentlicht werden.
\end{enumerate}

\section{Fristen}
Alle in dieser Satzung genannten Fristen beginnen jeweils an dem auf die Absendung des jeweiligen
Schreibens folgenden Tag oder am Tag nach der Beschlussfassung.

\section{Auflösung des Vereins}
\begin{enumerate}[label=(\arabic*)]
	\item  Die Auflösung des Vereins erfolgt, wenn sie vom Vorstand oder einem Drittel der Mitglieder
	      beantragt und von mindestens drei Viertel der anwesenden stimmberechtigten Mitglieder
	      beschlossen wird.
	\item  Die Auflösung kann nur von einer eigens zu diesem Zweck einberufenen Mitgliederversammlung
	      beschlossen werden. Die Mitgliederversammlung ist in diesem Fall nur beschlussfähig, wenn
	      mindestens die Hälfte aller Mitglieder erschienen ist. Ist dies nicht der Fall, ist innerhalb von drei
	      Wochen nach dem ersten Termin eine weitere Mitgliederversammlung einzuberufen, die ohne
	      Rücksicht auf die Zahl der erschienenen Mitglieder beschlussfähig ist.
	\item  Die Versammlung bestimmt zur Abwicklung der Geschäfte zwei Liquidator:innen, deren Aufgaben
	      und Befugnisse sich nach den Vorschriften des BGB richten.
	\item Bei Auflösung des Vereins oder Wegfall steuerbegünstigter Zwecke fällt das Vermögen zu
	      gleichen Teilen an die Stadt Rheine und den Kreis Steinfurt (Körperschaften des öffentlichen Rechts)
	      zwecks Verwendung für die Förderung
	      \begin{itemize}
		      \item der Wissenschaft und Forschung (§ 52 AO Abs. 2 Ziff. 1)
		      \item der Jugendhilfe (§ 52 AO Abs. 2 Ziff. 4)
		      \item der Kunst- und Kultur (§ 52 AO Abs. 2 Ziff. 5)
		      \item der Erziehung, Volks- und Berufsbildung (§ 52 AO Abs. 2 Ziff. 7)
	      \end{itemize}
\end{enumerate}

\section{Salvatorische Klausel}
\begin{enumerate}[label=(\arabic*)]
	\item Sollten einzelne Bestimmungen dieser Satzung gegen geltendes Recht verstoßen oder sich als
	      ungültig erweisen, so bleibt die Satzung im Übrigen gültig.
	\item Anstelle der ungültigen Bestimmung tritt eine Regelung, die dieser inhaltlich am nächsten kommt.
\end{enumerate}

\textbf{\textit{Vorliegende Fassung der Satzung wurde in der Mitgliederversammlung am 25.04.2024 beschlossen.}}
\end{document}