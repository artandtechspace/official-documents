\documentclass[12pt]{article}
\usepackage[a4paper, headheight=2.5cm, top=4.5cm, bottom=2.5cm, left=2.5cm, right=2.5cm]{geometry}
\usepackage[utf8]{inputenc}
\usepackage[T1]{fontenc}
\usepackage[ngerman]{babel}
\usepackage{graphicx}
\usepackage{enumitem}
\usepackage{fancyhdr}
\usepackage{hyperref}
\usepackage{microtype}
\usepackage{amsmath}
\usepackage{tocloft}
\usepackage{titlesec}
\usepackage{lastpage}
\usepackage{tabularx}
\usepackage{makecell}
\usepackage{ragged2e}
\usepackage{eurosym}

\renewcommand{\headrulewidth}{0pt}
\titleformat{\section}{\large\bfseries}{§ \thesection}{1em}{}
\renewcommand{\cftsecleader}{\cftdotfill{\cftdotsep}}

\title{Beitragsordnung des Vereins \\
ARTandTECH.space e.V.}
\author{
	Lindenstraße 11 \\
	48431 Rheine \\
    \
	}

\hypersetup{
  pdftitle={Beitragsordnung ARTandTECH.space e.V.},
  pdfauthor={ARTandTECH.space e.V.},
  pdfsubject={Beitragsordnung},
  pdfkeywords={Verein, Satzung, Beitragsordnung, Rheine},
  colorlinks=false,
  pdfborder={0 0 0},
  pdfstartview=Fit
}

\date{24. April 2024}

\begin{document}

% Titelseite
\maketitle
\thispagestyle{empty}

\begin{center}
	\includegraphics[width=0.3\textwidth]{assets/logo.png}
\end{center}

\begin{abstract}
	Wesentliche Grundlage für die finanzielle Ausstattung des Vereins ist das Beitragsaufkommen der Mitglieder.
	Der Verein ist daher darauf angewiesen, dass alle Mitglieder ihre Beitragspflichten, die in der Satzung grundsätzlich
	geregelt sind, in vollem Umfang und pünktlich erfüllen. Nur so kann der Verein seine Aufgaben erfüllen und seine
	Leistungen gegenüber seinen Mitgliedern erbringen.
\end{abstract}
\newpage

% Inhaltsverzeichnis
\pagestyle{fancy}
\fancyhf{}
\fancyhead[L]{\textbf{Beitragsordnung}}
\fancyhead[R]{\includegraphics[width=0.15\textwidth]{assets/logo.png}}
\tableofcontents
\thispagestyle{fancy}
\newpage

% Inhalt
\fancyfoot[R]{Seite \thepage\ von \pageref{LastPage}}
\setcounter{page}{1}

\section{Grundsatz}
Grundlage dieser Beitragsordnung ist § 6 der Vereinssatzung. Diese Beitragsordnung ist nicht Bestandteil der Satzung.
Sie regelt die Beitragsverpflichtungen der Mitglieder sowie die Gebühren und weitere Umlagen. Sie wird von der
Mitgliederversammlung des Vereins beschlossen.

\section{Beitragsjahr und Fälligkeiten}
Beitragsjahr ist das Kalenderjahr. Beiträge werden als Jahresbeitrag festgesetzt und jeweils zum 1. Februar eines Jahres fällig.
Mit Beginn der Mitgliedschaft wird der Jahresbeitrag für das laufende Jahr fällig.
Auf Antrag können Mitgliedsbeiträge in Raten gezahlt werden, über den Antrag entscheidet der Vorstand.

\section{Beiträge}
Die folgenden Beiträge sind Mindestbeiträge.

\begin{center}
	\begin{tabularx}{\textwidth}{|l|>{\RaggedRight\arraybackslash}X|c|}
		\hline
		\textbf{Beitragsklasse} & \textbf{Mitgliedsform}                                                                                                                                                         & \textbf{Beitragshöhe pro Jahr in EUR} \\
		\hline
		01                      & Schüler, Studierende und Auszubildende Sowie Freiwilligendienstleistende bis zum 25. Lebensjahr, Inhabende der Müunsterlandkarte, der Jugendleiterkarte und der Ehrenamtskarte & 24€                                   \\
		\hline
		02                      & Familien                                                                                                                                                                       & 80€                                   \\
		\hline
		02.1                    & Familien als Inhabende der Münsterlandkarte                                                                                                                                    & 30€                                   \\
		\hline
		03                      & Personen, welche nicht unter 01 und 02 fallen                                                                                                                                  & 60€                                   \\
		\hline
		04                      & Juristische Personen                                                                                                                                                           & 100€                                  \\
		\hline
	\end{tabularx}
\end{center}

Jedes Mitglied kann freiwillig einen höheren Beitrag zahlen.
Bei besonderen finanziellen Notlagen kann der Vorstand die Beiträge stunden oder erlassen.

\section{Aufnahmebeitrag}
Ein Aufnahmebeitrag wird nicht erhoben.

\section{Mahn- und Versäumniskosten}
Zahlungen müssen spätestens am nächsten Werktag nach Fälligkeit auf dem Konto des Vereins eingegangen sein. Ist die Zahlung zu diesem Zeitpunkt beim Verein nicht eingegangen, befindet sich das Mitglied mit seiner Zahlungsverpflichtung in Verzug. Der ausstehende Beitrag wird dann mit einem Säumniszuschlag von 1,00 \euro{} je 7 Kalendertage belegt.
Weist das Konto eines Mitglieds zum Zeitpunkt der Abbuchung des Beitrages \slash der Gebühren \slash der Umlage keine Deckung auf, so haftet das Mitglied dem Verein gegenüber für sämtliche dem Verein mit der Einziehung des Beitrages (einschl. Kosten für Rücklastschriften) verbundenen Kosten. Dies gilt auch für den Fall, dass ein bezogenes Konto erloschen ist und das Mitglied dies dem Verein nicht mitgeteilt hat.
Muss ein Mitglied wegen fehlender Beitragszahlungen schriftlich zur Zahlung aufgefordert werden, werden folgende Mahnkosten fällig:

\begin{center}
	\begin{tabularx}{\textwidth}{|l|>{\RaggedRight\arraybackslash}X|c|}
		\hline
		\textbf{Mahnstufe} & \textbf{Mahnkosten in EUR} & \textbf{Frist} \\
		\hline
		1.                 & 5,00€                      & 4 Wochen       \\
		\hline
		2.                 & 10,00€                     & 6 Wochen       \\
		\hline
	\end{tabularx}
\end{center}

\textbf{Bei geschuldeten Mitgliedsbeiträgen wird im Anschluss ein Verfahren zur Streichung von der Mitgliederliste eingeleitet. Offene Forderungen bleiben weiterhin bestehen.}

\section{Umlagen, Sonderbeiträge}
Derzeit sind diese nicht festgesetzt.

\section{Zahlungsmodalitäten}
\begin{enumerate}[label=(\arabic*)]
	\item Für die Beitragshöhe ist die am Fälligkeitstag bestehende Mitgliedsform maßgebend. Die
	      Mitgliedsform wird beim Eintritt in den Verein erstmalig festgelegt. Änderungen derselben
	      sind von den Mitgliedern unter Vorlage geeigneter Nachweise zu beantragen.
	\item Änderungen der persönlichen Angaben sind schnellstmöglich mitzuteilen.
	\item Mitgliedsbeiträge, Gebühren und Umlagen werden im SEPA-Basis-Lastschriftverfahren
	      eingezogen. Das Mitglied hat sich hierzu bei Eintritt in den Verein zu verpflichten, ein SEPA- Lastschriftmandat zu erteilen sowie für eine ausreichende Deckung des bezogenen Kontos zu sorgen.
	      Der Mitgliedsbeitrag wird unter Angabe einer Gläubiger-ID und der Mandatsreferenz jährlich zum 1. Februar eingezogen. Fällt dieser nicht auf einen Bankarbeitstag, erfolgt der Einzug am unmittelbar darauf folgenden Bankarbeitstag.
	\item Das Mitglied hat für eine pünktliche Entrichtung des Beitrages, der Gebühren und Umlagen Sorge zu tragen. Gebühren, Umlagen und sonstige Beiträge werden unter Bekanntgabe eines besonderen Datums fällig und eingezogen. Zwischen der Bekanntgabe und dem Einzug liegen mindestens 28 Kalendertage
	\item DerVorstandistermächtigt,dieZahlungvonBeiträgenaufAntragzustunden,zu ermäßigen, zu erlassen oder eine Ratenzahlung zu vereinbaren. Ein Rechtsanspruch auf Ratenzahlung und / oder Stundung der Beitragsschuld besteht nicht.
	\item Erfolgt der Vereinseintritt nach dem 30.06. erfolgt eine Berechnung von 50\% des Beitragssatzes.
\end{enumerate}

\section{Vereinskonto}

\begin{center}
	ARTandTECH.space e.V.
\end{center}
\begin{center}
	IBAN: DE75 4035 0005 0000 0632 97
\end{center}
\begin{center}
	BIC: WELADED1RHN
\end{center}
\begin{center}
	Kreditinstitut: Sparkasse Rheine
\end{center}
Überweisung auf andere Konten sind nicht zulässig und werden nicht als Zahlungen anerkannt.

\section{Vereinsaustritt}
Der Austritt aus dem Verein kann nur unter Beachtung des § 5 der Satzung erfolgen. Mitgliedsbeiträge, die im Jahr des Austritts eingezogen wurden, werden nicht erstattet.

\begin{center}
	\textbf{\textit{Vorliegende Fassung der Beitragsordnung wurde in der Mitgliederversammlung am 25.04.2024 beschlossen.}}
\end{center}
\end{document}